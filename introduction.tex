\chapter{Introduction}
Datatype-generic operations are operations that are defined over the shapes
of objects rather than any specific types of objects. This is different from
the concept of parametric polymorphism, also called generic programming in
languages like Scala:

Where parametric polymorphism abstracts away the elements of a container,
datatype-generic programming essentially abstracts away the container.

In the rest of this thesis, `generic programming' means `datatype-generic
programming'.


\begin{example}[Parametric polymorphism]
The type constructor \cd{List} is an example for parametric polymorphism. For
example, you can generate the length of a list:

\begin{lstlisting}
    def length[T](list: List[T]) = ...
\end{lstlisting}
\end{example}

\begin{example}[Datatype-generic operation]
A datatype-generic operation is an operation that abstracts the outer type, and
works with specific inner types. For example, a sum that can work on containers
of integers; that is, something like:
\begin{lstlisting}
    def sum[C[_]](list: C[Int]) = ...
\end{lstlisting}
\end{example}

There are two basic ways of how datatype-generic programming libraries expose
their interfaces:

\begin{enumerate}
    \item By representing objects as sums and products
    \item By providing combinators for performing operations
\end{enumerate}

In the first style, a generic operation will traverse the data structure and
perform its operation. In the second style, the generic library takes care of
the traversal and offers some combinators to allow working with the elements.
As a comparison: The first style is similar to traversing a list manually whereas
the second style resembles a fold.

\section*{What defines a good generic programming library?}

In combinator-based libraries, there are usually three different categories of
combinators (where \cd{T} is a fixed type):

\begin{enumerate}
    \item \textbf{Consumers} take a generic object and return an object of a fixed type, they have
           a type of the form \cd{a => T}

           Example: The sum function from above.

    \item \textbf{Transformers} map from one generic type to another or the same, the have a type of the form
            \cd{a => a} or \cd{a => b}


            Example: A function that increments every integer in an object, returning the modified one.

    \item \textbf{Producers} produce a generic object, they have a type of the form: \cd{T => a}

            Example: A function that takes an integer and the representation of
                     a tree and returns a tree.
\end{enumerate}


A powerful generic programming library should support the following criteria:

\paragraph{Ad-hoc cases:} Can a function deal with a specific type in an ad-hoc way?

For example, if the library exposes sums and products; can a function
only work on sums and products, or can it match a type directly, for
example; can it find all lists in an object?

\paragraph{Extensibility:} Can an existing function be extended with new ad-hoc cases?

For example, imagine a pre-existing function to calculate the depth of a data
structure. If you write a new tree type that stores the depth in its nodes, can
you extend the pre-existing function to simply return that value?

\paragraph{Multi-parametric functions:} Can there be generic functions with two generic parameters?

For example, you might want to define a generic equality function.

\paragraph{First-class generics:} Can a generic function be passed to another function?

For example, suppose a function \cd{gmapQ} that takes another generic function \cd{f}
and an object and maps the function \cd{f} over all members of the object, constructing
a list of new objects.

\paragraph{Automatic representations:} Generic libraries are usually implemented
using some sort of representation objects that describe the shape of the
objects of a type. Is it possible to create those automatically?

\paragraph{Separate compilation:} Can I introduce a new type in a new module
and use it with a generic function in an existing module?

\textit{All of the libraries in this thesis support this.}

\paragraph{Constructor names:} If objects are represented as sums of products;
can those products have names?

For example, a pretty printer would want to print constructor names.


\section*{Current state}
Generic programming is primarily used by the Haskell community. There are
several libraries available; and the Glasgow Haskell Compiler already provides
a library as part of its \cd{base} package, the \cd{Data.Data} module. This
module is based on another library called `Scrap your boilerplate'\cite{DBLP:conf/tldi/LammelJ03}
and is combinator-based.

Scala has one generic programming library called `Shapeless'. Shapeless was
initially derived from scrap your boilerplate, but became more sophisticated
over the time, adding features like heterogeneous lists.
