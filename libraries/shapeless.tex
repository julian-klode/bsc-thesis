\section{Shapeless}
\label{shapeless}

Shapeless is an already existing Scala library that started as a port of Scrap your
Boilerplate~\cite{DBLP:conf/tldi/LammelJ03} from Haskell. In the mean time however,
it evolved a bit further and now covers topics like heterogeneous lists and
zippers, among other things.

Compared to the LIGD, EMGM, and Shapeless ports, Shapeless is far more
advanced. It automatically works on custom Scala types without having to
write instances of any type class, and it also uses several advanced programming
techniques like dependent types for its implementation. This means that using
Shapeless is easy, as long as there is no error. Debugging errors on the other
hand can be quite complicated.

\subsection{Polymorphic functions}
Shapeless provides support for writing polymorphic functions in the
\cd{shapeless.poly} package.

\begin{example}[Extracting integers]
  The function \cd{int1} returns a passed integer and 1 (the identity of
  multiplication) for all other types.

  \lstinputlisting[firstline=52,lastline=57,label=src:shapeless:int1,caption=\cd{int1} -- A polymorphic function of arity 1]{scala/src/shapeless.scala}
\end{example}

\begin{example}[Lifting a monomorphic function to a polymorphic one]
  The class \cd{->} of the \cd{shapeless.poly} package takes as an argument a monomorphic function
  and lifts it to a \cd{Poly1} instance. Using it is simple:

  \lstinputlisting[firstline=67,lastline=67]{scala/tests/CompanyTest.scala}
\end{example}

\begin{example}[Calculating products of numbers]
  The function \cd{prod} calculates the products of numbers. Given two \cd{Int}
  objects it produces an \cd{Int}, given two \cd{Float} it produces a \cd{Float},
  and so on.
  \lstinputlisting[firstline=52,lastline=57,label=src:shapeless:prod,caption=\cd{prod} -- A polymorphic function of arity 2]{scala/src/shapeless.scala}
\end{example}

\subsection{The \cd{everything} combinator}
Shapeless provides a combinator called \cd{everything} for constructing
queries on objects. This takes two polymorphic functions: The first one
extracts objects we are interested in, the second one combines those objects
to a result -- it basically is a fold.

\begin{example}[Calculating the product of all integers in an object]
  If we combine \cd{int1} and \cd{prod} from the previous subsection we can
  easily calculate the product of all integers in a data structure (recursively).
  \lstinputlisting[firstline=73,lastline=73,label=src:shapeless:product,caption=\cd{product} -- Product of all integers in an object]{scala/src/shapeless.scala}
\end{example}

\subsection{The \cd{everywhere} combinator}
The \cd{everywhere} combinator implements transformations. It takes a
polymorphic function and applies that to a data structure, types
not handled by it will be recursed into.

\begin{example}[Increasing salaries]
  Applying \cd{everywhere} to the function \cd{incSalary} introduced earlier
  in that chapter yields a new function that can be applied to any object
  and will increment all salaries within that object, returning the changed
  object.

  \begin{lstlisting}
    scala> everywhere(incSalary)(1)
    res0: Int = 1
    scala> everywhere(incSalary)(List(Salary(10f)))
    res1: List[CompanyData.Salary] = List(Salary(11.0))
  \end{lstlisting}
\end{example}

\subsection{Heterogeneous lists (\cd{HList})}
Heterogeneous lists are like a blend of tuples and lists. That is, they
consist of heterogeneous elements and are variable size. Their type encodes
the length and the type of the elements, and operations like prepending calculate
the result type through the use of dependent types.
A heterogeneous list is created by prepending elements to another heterogeneous
list using the \cd{::} operator. The empty list is the object \cd{HNil}.

\begin{example}[A heterogeneous list] {\ } % force a line
\begin{lstlisting}
  scala> 1 :: "Hello, world" :: HNil
  res0: shapeless.::[Int,shapeless.::[String,shapeless.HNil]] = 1 :: Hello, world :: HNil
\end{lstlisting}
\end{example}

\subsection{Zippers}
Shapeless provides an implementation of generic zippers. A generic zipper is
a data structure that represents a moment in the traversal of another data
structure like a tree. It has operations like \cd{left}, \cd{right}, \cd{down},
and \cd{up} for navigating through the object and \cd{get} to get the object
at the current position.

\begin{example}[Object to zipper]
Importing the \cd{syntax.zipper._} sub-package adds a \cd{toZipper} method to objects.

\begin{lstlisting}
  scala> ((1,1),2,3).toZipper
  res0 = Zipper(HNil,(1,1) :: 2 :: 3 :: HNil,None)
  scala> res0.down.get
  res1: Int = 1
\end{lstlisting}
\end{example}

\subsection{Further types and operations}
Shapeless also provides several additional types and operations, such as
natural numbers on the type level, heterogeneous maps, and more.
