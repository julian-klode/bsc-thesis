\section{LIGD$^{+}$}
\summary{The straight-forward solution to most problems}
\lstinputlisting[firstline=53,lastline=68,float=ht,caption=LIGD Basics,label=ligd-base]{scala/src/ligd.scala}

LIGD\cite{Cheney:2002:LIG:581690.581698} is a simple library that provides a \cd{Rep} trait that describes
the structure of a type using sums, products, and scalar values. This
version of LIGD is more powerful than the original one, though, as we
will see later.

Listing~\ref{ligd-base} shows the basic \cd{Rep} trait with instances for
scalar types, sums (represented using \cd{Either}), and products (represented
using \cd{Tuple2}). As explained in the previous chapter, we use implicit
here to allow the compiler to automatically infer the correct \cd{Rep}
instance for generic function calls.


\subsection{User-defined types}
In order to work with user-defined (case) classes, those need to be
translated to a representation using sums and products. We need an
isomorphism between the user-defined class and a sum of product type. First
of all, we can define an isomorphismus as:
\begin{lstlisting}[gobble=2]
  class EP[B, C](val from: B ⇒ C, val to: C ⇒ B)
\end{lstlisting}
Now we can define a \cd{Rep} instance, that represents values of a custom
type \cd{B} and converts it to a sum of product notation of type \cd{C},
something like this:
\begin{lstlisting}[gobble=2]
  case class RType[C, B](val c: Rep[C], val ep: EP[B, C]) extends Rep[B]
\end{lstlisting}
There is one problem though: The representation of \cd{C} may be recursive,
for example a list type is a sum of nothing, and a product of an element and
the list type. Thus, the parameter \cd{c} must be lazy. Scala does not allow
lazy \cd{val}s on case classes, though, so this needs to be worked around.

There are two approaches: One is encoding the thunk manually, like:
\begin{lstlisting}[gobble=2]
  case class RType[C, B](val c: () => Rep[C], val ep: EP[B, C]) extends Rep[B]
\end{lstlisting}
This is very easy, but looks inconsistent in usage. Another approach is to
not use a case class and provide a companion object with custom \cd{apply}
and \cd{unapply} methods. Listing~\ref{ligd-rtype} shows how this is
implemented.
\lstinputlisting[firstline=76,lastline=90,float=ht,caption=Implementation of RType,label=ligd-rtype]{scala/src/ligd.scala}

\begin{example}[Lists]
In order to represent a list, we encode the
list as a sum of products, or to be precise, we want to encode a \cd{List[A]}
as: \cd{Either[Unit, (A, List[A])]}. In order to do this, we need functions
that convert from \cd{List[A]} to \cd{Either[Unit, (A, List[A])]} and
vice versa. An implementation of those functions is given in figure~\ref{ligd-rlist}.
%-- We can then define an RType instance as:
%\lstinputlisting[firstline=104,lastline=108]{scala/src/ligd.scala}

\lstinputlisting[firstline=96,lastline=117,float=ht,caption=Implementation of rList,label=ligd-rlist]{scala/src/ligd.scala}

Note how the first parameter of \cd{RType} is recursive.

There is one issue with this representation though: It leads to deep recursion,
potentially causing a stack overflow in functions using it.
\end{example}

\subsection{Writing generic functions}
Writing generic functions using LIGD is simple. We just pass the object(s)
we are interested in to the function, along with their type representations,
and then pattern match on the type representations.

\begin{example}[Generic equality]
Listing~\ref{ligd-geq} shows how to implement a simple function that checks for
generic equality.

  \lstinputlisting[firstline=36,lastline=51,float=ht,caption=Generic Equality in LIGD,label=ligd-geq]{scala/src/ligd.scala}

This function does not deal with cycles nor does it support very long lists. In
the first case, it would need to keep track of which objects it visited already;
and in the second case, it would need to manage stack itself or be tail recursive.

Making the function tail-recursive seems straight forward. There is one issue,
though: Scala will not consider the calls to \cd{geq} eligible for tail call
optimization, as the type parameters differs. The only ways to deal with deeply
nested structures are thus manual stack management or trampolines\footnote{The
\cd{scala.util.control.TailCalls} object makes trampolines easy to write}. This
does not matter much in this chapter.
\end{example}


\subsection{Differences and Extensions to the original Haskell implementation}
One minor difference is that we implemented \cd{Rep} like we would implement
a type class in Haskell. The original paper manually passed around \cd{Rep}
instances for generic functions, we make use of Scala's implicit values.


\subsubsection{Ad-hoc cases using subtyping}
A key difference that extends the functionality of LIGD in Scala is that
Scala permits sub-typing. If we look at our definition of RType, we see
that it is not \cd{sealed}, allowing it to be extended by subclasses.

We can use this to create custom sub classes for types we are interested
in, and pattern match on them. Because those are sub classes of \cd{RType},
existing code will be unaffected by new types.

An example for this is the list type. Because we extended \cd{RType} as
\cd{RList} we can now write a generic function that finds all lists of a
specific element type in a data structure.

\subsection{Extensibility}
\label{lidg-extensible}
Imagine we have an existing generic function that we do not control (for example,
in a different module) and want to add a special case to that function. Because
the existing generic function has self-recursion this is usually impossible.

It turns out that if we change the way we write the generic function slightly,
we can actually extend it with new special cases. The changes are very simple:

\begin{enumerate}
  \item Convert the function into a class with an apply method containing the
        existing function and replace all recursive calls to the functions with
        calls to apply.
  \item Provide an object that extends that class, this can then be called like
        a function.
\end{enumerate}

We can now easily extend that `function', by extending the class and overriding
the apply method, matching on the cases we are interested in and deferring the
remaining cases to the implementation in the super class.

For example, if we transform \cd{geq} according to those rules, we can
easily extend it with a special case for lists, as shown in listing~\ref{src:ligd-extensible:geqlist}.\footnote{The complete example with the transformed \cd{geq} can be found in
section~\ref{src:ligd-extensible}}
\lstinputlisting[firstline=76,lastline=85,float=ht,caption=Extended Generic Equality in LIGD,label=src:ligd-extensible:geqlist]{scala/src/ligd-extensible.scala}
This approach using classes is not applicable to Haskell, but it is possible
to do something similar - by passing the function as an argument:
\begin{enumerate}
  \item Convert \cd{geq :: Rep[a] -> a -> a -> Bool} to \\
        \hphantom{Convert }\cd{geq' :: Rep[a] -> a -> a -> (forall a. Rep[a] -> a -> a -> Bool) -> Bool} \\
        and change self-recursive calls to call the new argument
  \item Add a  \cd{geq rep a b = geq' rep a b geq}
\end{enumerate}

Extending the function should now be straight forward. It is less useful to
have this in Haskell, though, because LIGD in Haskell does not support ad-hoc
cases, meaning we can only override the pre-defined type representations and not
match on custom types. If support for named constructors is added however, it
could still become useful enough.\footnote{See \url{http://code.haskell.org/generics/comparison/LIGD/LIGD.lhs} for an LIGD with named constructors}

\subsubsection{Equality comparisons of type representations}
The Haskell version of LIGD does not support comparing Rep instances
against each other. This is quite useful, though, as it allows to
implement simple variants of generic operations that look for objects
of a specific type in another object, such as folds.

For this to work, it is important to always use values for \cd{Rep} instances
for non-parameterized types, so that they compare correctly without a custom
\cd{equals()} method.
When representing parameterized types such as Lists, we can create a subclass
of RType with Rep parameters for the parameters of the type. We can then
define a custom \cd{equals()} method, as seen in listing~\ref{rmycustomtype}.

\begin{lstlisting}[float,caption=Pattern for representing parameterized types,label=rmycustomtype]
  class RMyCustomType[T1, ...](r1: Rep[T1], ...) {
    override def equals(b: Any) : Boolean = b match {
      case RMyCustomType(r1, ...) => r1 == this.r1 &&
                                     r2 == this.r2 && ...
      case _ => false
    }
  }
\end{lstlisting}
