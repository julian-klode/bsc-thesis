\section{EMGM -- Translation of Extensible and Modular Generics for the Masses}
\summary{More verbose (especially for ad-hoc cases) than LIGD}


In Generics for the Masses \cite{GM}, Ralf Hinze introduces an alternative
encoding of generic functions. While objects are still viewed as sums and
products of scalar values, generic functions are now defined as instances
of a \cd{Generic} type class, with one method for each type of object
supported, our slightly extended version can be seen in Figure~\ref{src:emgm:Generic}
\lstinputlisting[firstline=40,lastline=50,label=src:emgm:Generic,caption=The \cd{Generic} trait in GM]{scala/src/emgm.scala}
The parameter \cd{G} refers to a type constructor for a type that stores
a function of a specific type.

\begin{example}[Generic equality]
Imagine we want to check arbitrary objects of the same type for equality.
Our type \cd{G[\_]} must have a function that accepts two
value of the type given by the first parameter, and returns a boolean
-- see Listing~\ref{src:emgm:GEq}.
\lstinputlisting[firstline=172,lastline=192,float=ht,label=src:emgm:GEq,caption=A generic equality function in GM]{scala/src/emgm.scala}

We use closures here instead of normal methods in order to have the
compiler infer the types automatically, as explained in Section~\ref{type-inference}. \qed
\end{example}

Being able to represent generic functions is only half of the work, though
-- we still need a way to represent objects. Instead of converting objects
to a representation of sums and products, we simply call the members of
the \cd{Generic} instance we need.

\lstinputlisting[firstline=58,lastline=82,float=ht,label=src:emgm:Rep,caption=The \cd{Rep} trait in GM]{scala/src/emgm.scala}

We can now, similar to  LIGD, represent arbitrary types as sums of products.
Recall the list example from LIGD.
As we can see in Figure~\ref{src:emgm:rList}, the definition is similar to
the one in LIGD
-- the isomorphism is basically the same, the only difference is the slightly
more verbose definition of \cd{RList}, because we need to split the generic
function `creation' (the description of the type) out in \cd{rList}.

We will see that this only gets worse once we introduce ad-hoc cases.
\lstinputlisting[firstline=119,lastline=138,float=ht,label=src:emgm:rList,caption=Representing lists in GM]{scala/src/emgm.scala}

\subsection{Extensibility and Modularity}
Now consider that we want to treat lists specially, that is, create an
ad-hoc case of \cd{geq} that handles lists directly, instead of as a
sum of products representations. That is, we need a subclass of Generic
that can work on lists, something like:
\lstinputlisting[firstline=240,lastline=242,label=src:emgm:GenericList,caption=Generic with ad-hoc list case]{scala/src/emgm.scala}
Note that the special case for lists defers to the sum of product variant
if it is not handled by the generic function. We can thus use that extra
method (override it) or ignore it (and treat it using sums and products).
\lstinputlisting[firstline=253,lastline=270,float=ht,label=src:emgm:GEqList,caption=Generic equality specialised to lists]{scala/src/emgm.scala}

What we are still missing is a sort of dispatcher, like \cd{Rep} in Generics
for the masses. Without such a dispatcher, using generic functions like
\cd{GEqList} would not be easy.

\subsection{Comparison to LIGD}
