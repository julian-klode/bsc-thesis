\section{EMGM}
\summary{More verbose (especially for ad-hoc cases) than LIGD}


In Generics for the Masses~\cite{GM}, Ralf Hinze introduces an alternative
encoding of generic functions. While objects are still viewed as sums and
products of scalar values, generic functions are now defined as instances
of a \cd{Generic} type class, with one method for each type of object
supported, a slightly extended and translated version can be seen in Figure~\ref{src:emgm:Generic}.
The parameter \cd{G} refers to a type constructor of a type that stores
a function of a specific type.
\lstinputlisting[firstline=41,lastline=51,label=src:emgm:Generic,caption=The \cd{Generic} trait in GM]{scala/src/emgm.scala}


In addition to the cases known from LIGD, this also introduces a \cd{constr}
case for representing named constructors of a certain arrity. The default
implementation of \cd{constr} will just return the underlying \cd{G[A]} and
ignore the name and arrity, but functions are free to override it -- for example,
a generic show function might use this to pretty print data types.

\begin{example}[Generic equality]
Imagine we want to check arbitrary objects of the same type for equality.
Our type \cd{G[\_]} must have a function that accepts two
values of the type given by the first parameter, and returns a boolean
-- see Listing~\ref{src:emgm:GEq}.
\lstinputlisting[firstline=173,lastline=193,label=src:emgm:GEq,float=ht,caption=A generic equality function in GM]{scala/src/emgm.scala}

We use closures here instead of normal methods in order to have the
compiler infer the types automatically, as explained in Section~\ref{type-inference}.
As with LIGD, deeply nested data structures and cycles will cause this
function to fail. \qed{}
\end{example}

Being able to represent generic functions is only half of the work, though
-- we still need a way to represent objects. Instead of converting objects
to a representation of sums and products, we simply call the members of
the \cd{Generic} instance we need. Some of the instances of such a \cd{Rep}
trait are shown in listing~\ref{src:emgm:Rep}.

\lstinputlisting[firstline=59,lastline=71,label=src:emgm:Rep,caption=The \cd{Rep} trait in GM and some instances]{scala/src/emgm.scala}

\begin{example}[Representing lists]
We can now, similar to  LIGD, represent arbitrary types as sums of products.
Recall the list example from LIGD.
As we can see in Figure~\ref{src:emgm:rList}, the definition is similar to
the one in LIGD
-- the isomorphism is basically the same, the only difference is the slightly
more verbose definition of \cd{RList}, because we need to split the generic
function `creation' (the description of the type) out in \cd{rList}.

We will see that this only gets worse once we introduce ad-hoc cases.
\lstinputlisting[firstline=120,lastline=139,float=ht,label=src:emgm:rList,caption=Representing lists in GM]{scala/src/emgm.scala}
\qed{}
\end{example}

\subsection{Extensibility and Modularity}
Now consider that we want to treat lists specially, that is, create an
ad-hoc case of \cd{geq} that handles lists directly, instead of as a
sum of products representations. That is, we need a subclass of Generic
that can work on lists, something like:
\lstinputlisting[firstline=239,lastline=241,label=src:emgm:GenericList,caption=Generic with ad-hoc list case]{scala/src/emgm.scala}
Note that the special case for lists defers to the sum of product variant
if it is not handled by the generic function. We can thus use that extra
method (override it) or ignore it (and treat it using sums and products).

\begin{example}[Generic equality extended for lists]
An example for using it would be a generic equality function with an optimized
version for lists, like figure~\ref{src:emgm:GEqList} shows.
\lstinputlisting[firstline=252,lastline=269,label=src:emgm:GEqList,caption=Generic equality specialised to lists]{scala/src/emgm.scala}
If we do not override \cd{list}, it would simply use the isomorphism. \qed
\end{example}

What we are still missing is a sort of dispatcher, like \cd{Rep} in `Generics
for the masses', but this time, \cd{rep} should not be parameterized over \cd{G},
rather the type itself should be -- the resulting \cd{GRep} trait defines a
representation of a type for a specific generic function, rather than for all
generic functions.
\lstinputlisting[firstline=201,lastline=203,label=src:emgm:GRep,caption=The \cd{GRep} dispatcher]{scala/src/emgm.scala}

The instances of \cd{GRep} are basically the same as those of \cd{Rep}, except
that they capture the \cd{Generic[G]} in the constructor and not in the
method.

\lstinputlisting[firstline=233,lastline=236,label=src:emgm:GRProd,caption=A \cd{GRep} instance for products]{scala/src/emgm.scala}

We can now add a representation for our special case of lists:

\lstinputlisting[firstline=244,lastline=247,label=src:emgm:GRList,caption=A \cd{GRep} instance for lists]{scala/src/emgm.scala}

Unlike the other instances of \cd{GRep}, this one does not take a \cd{Generic[G]},
but rather a \cd{GenericList[G]}. This means that functions will only work on
lists if they extend \cd{GenericList}. This is easy to do, though: Because
\cd{GenericList} provides a default implementation of \cd{list} that defers to
the view of sum of products, an empty declaration is sufficient. For example,
instead of the special case for lists in \cd{GEqList}, we could just use:
\begin{lstlisting}
    implicit object GEqList extends MyGEq with GenericList[GEq]
\end{lstlisting}
to extend the universe of \cd{geq} to lists -- there is no need to actually use any
of the special cases.
