\section{Uniplate}
\summary{Not feasible without macros or code generators and fails generic equality test}

Uniplate is a library that differentiates between two basic type of
objects: Uniplates and Biplates. Uniplates are simple recursive types
like terms or expressions, which form a graph where all inner nodes have
the same common supertype. Biplates are types that contain other Uniplates,
for example, a \cd{List a} has a \cd{Biplate List a}.

\subsection{Uniplates}
The \cd{Uniplate} type class, shown in listing~\ref{src:uniplate:Uniplate}
describes recursive types like terms and expressions. Its only primitive
operation is the \cd{uniplate()} method.

\lstinputlisting[firstline=38,lastline=56,label=src:uniplate:Uniplate,caption=The Uniplate type class (and various additional functions)]{scala/src/uniplate.scala}

The \cd{uniplate()} method takes an object of type \cd{T} and returns a tuple
of all direct children of the same type and a function that constructs a new \cd{T}
from a list of children of the same type.


There are also several more functions that can be defined on top of \cd{uniplate()}.

The \cd{children()} function returns a list of all direct children of the same
type. For example, in a term, this will be the operands.

The \cd{universe()} function returns a list of all transitive children of the
same type. For example, in a term, this will be the operands and their children,
and so on. In a tree, applied to the root, this will be all nodes in the tree.

The \cd{transform()} function defines a bottom up transformation. It applies
the passed function recursively to all children bottom-up.

\begin{example}[Transformation]
An example of a bottom-up transformation would be eliminating subtraction in expressions, that
is transform \cd{a - b} to \cd{a + (-b)}. Given an existing type \cd{Expr}
and \cd{Add}, \cd{Sub}, \cd{Neg} case classes of it, such a transformation
could be written as:

\begin{lstlisting}
    def removeSub(e: Expr) = {
        case Sub(a, b) => Add(a, Neg(b))
        case e => e
    }

    val expressionWithoutSub = transform(removeSub)(expression)
\end{lstlisting}
\end{example}

The \cd{descend()} function applies a top-down transformation.

The \cd{rewrite()} function applies a transformation until some sort of
normal form has been reached. The passed function returns a \cd{Some} if
it applied a transformation to an object and a \cd{None} if the input was
in normal form already.

In the original uniplate paper\cite{DBLP:conf/haskell/MitchellR07}, Mitchell
and Runciman introduce additional functions, but those are not implemented
in the Scala version, because the focus was on biplates.

\subsection{Biplates}
Simply speaking, biplates are containers of uniplates. For example, a type
\cd{Expr} that represents arithmetic expressions over integers has a corresponding
\cd{Biplate[Expr, Int]}.

\lstinputlisting[firstline=99,lastline=103,label=src:uniplate:Biplate,caption=The Biplate type class (and various additional functions)]{scala/src/uniplate.scala}

The basic operation of a \cd{Biplate} is the \cd{biplate} method. This method
is similar to the \cd{uniplate} method of the \cd{Uniplate} type class, but now
works on children of a different type.

The \cd{universeBi} and \cd{transformBi} methods are like their non-bi counterparts
in \cd{Uniplate}. Outlook: \cd{transformBi} corresponds to the
\cd{everywhere} combinator of Scrap your Boilerplate\cite{DBLP:conf/tldi/LammelJ03},
with one caveat: the passed function is not polymorphic.

\begin{example}[Minimum integer]
A function that calculates the minimum integer in a biplate can be defined
by folding \cd{scala.math.min} over the universe of the biplate.

\lstinputlisting[firstline=110,lastline=111,label=src:uniplate:minInt,caption=Uniplate minInt]{scala/src/uniplate.scala}
\end{example}

\subsection{Limitations and Advantages compared to other solutions}
Uniplate can theoretically be ported from Haskell to Scala, but it is not
feasible to do without a way to automatically generate instances of the
\cd{Biplate} type class, because the amount of biplates needed is quadratic
to the number of types.

Due to the way uniplate works, it is not possible to implement a generic
equality function for two biplate objects. Because biplates work by only
looking at a subset of the types elements (for example, a \cd{Biplate[Company,Salary]}
only cares about Salaries in a company -- not enough to compare two companies). Even
for Uniplates, defining a generic equality function won't work because we
only have access to the objects themselves and not to members of a different
type.

On the other hand, if Uniplate is applicable it will almost always lead to
the shortest code compared to other frameworks. See chapter~\ref{libraries}
for more examples.
