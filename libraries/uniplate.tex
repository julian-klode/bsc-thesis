\section{Uniplate}
\summary{Not feasible without macros or code generators and fails generic equality test}
\todo{Explain how uniplate works}

Uniplate is a library that differentiates between two basic type of
objects: Uniplates and Biplates. Uniplates are simple recursive types
like terms or expressions, which form a graph where all inner nodes have
the same common supertype. Biplates are types that contain other Uniplates,
for example, a \cd{List a} has a \cd{Biplate List a}.


Uniplate can theoretically be ported from Haskell to Scala, but it is not
feasible to do without a way to automatically generate instances of the
\cd{Biplate} type class, because the amount of biplates needed is quadratic
to the number of types.

Due to the way uniplate works, it is not possible to implement a generic
equality function for two biplate objects. Because biplates work by only
looking at a subset of the types elements (for example, a \cd{Biplate[Company,Salary]}
only cares about Salaries in a company -- not enough to compare two companies). Even
for Uniplates, defining a generic equality function won't work because we
only have access to the objects themselves and not to members of a different
type.

A basic implementation of uniplate can be found in the \cd{uniplate.scala}
source file in section~\ref{src:uniplate}.
