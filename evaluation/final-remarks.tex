\section{Final remarks}
Shapeless and LIGD are the best options for generic programming in Scala
right now. Shapeless provides a much larger feature set, and is also more
widely used, and should thus be preferred in production use.

The extended LIGD is now capable of supporting ad-hoc cases and extensibility,
making it more powerful then the original implementation in Haskell. Its use
of only basic language features and its easy to write representations and
functions make it an optimal choice for introducing generic programming to
students.

It would be interesting to look at automatic generation of LIGD representations
used macros; likewise, an implementation that represents constructors using
heterogeneous lists also seems a worthwhile idea, opening up new possibilities
like generic zippers. While an initial prototype of such a library exists in
the source code of this bachelor thesis\footnote{See \texttt{src/hligd.scala} and
\texttt{src/hlists.scala}}, it requires more work to make it work completely.

Another topic to look at with regards to LIGD is the handling of lists and
other container types: Representing them as deeply-nested pairs can cause
stack overflows when implementing algorithms in a naive way. Adding direct
support for sequences and/or implementing a set of combinators representing
commonly-needed operations using trampolines might be worthwhile.

Note that it is not possible to write simple tail-recursive generic functions in Scala,
because Scala considers the same function with different type parameters
to be a different function and will thus not be able to optimize those
tail-recursive calls into loops; requiring the use of trampolines instead.
