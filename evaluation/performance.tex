\section{Performance benchmark results}

\begin{table}[ht]
\begin{tabular}{c|r|r|r|r}
\bf{test}  & \bf{company}    & \bf{geq}        & \bf{min}        & \bf{sum}       \\
\hline
Direct     &          3,0 μs &          0,4 μs &          0,6 μs &          0,4 μs\\
Shapeless  &         39,3 μs &             N/A &          4,7 μs &          4,8 μs\\
LIGD       &         16,0 μs &         15,3 μs &         13,5 μs &         13,8 μs\\
EMGM       &         18,4 μs &         16,5 μs &         12,2 μs &         16,5 μs\\
\end{tabular}

\caption{Benchmark results}
\label{bench}
\end{table}

The performance of the libraries was tested by running various tests on them,
the measurements are done using scalameter 0.6 with 150 runs, default warmer,
and a \cd{Measurer.IgnoringGC with Measurer.OutlierElimination} as the measurer.

\begin{table}[ht]
\begin{tabular}{l|l}
    Scala & 2.11.2 \\
    JRE   & OpenJDK Runtime Environment (IcedTea 2.5.2) (7u65-2.5.2-4) \\
    JVM   & OpenJDK 64-Bit Server VM (build 24.65-b04, mixed mode) \\
    Arguments & -J-Xss1024m (increased stack size to 1GB) \\
    OS    & Debian GNU/Linux `unstable' (2014-09-21), kernel 3.16.2 \\\hline
    CPU   & Intel Core i5-3320M (2.6 GHz, 3.3 GHz turbo boost) \\
    RAM   & 8 GB
\end{tabular}
\caption{Benchmark environment}
\end{table}

The tests that were run were:

\begin{description}
    \item[company]The `company' test increases the salary in a company
object. The company consists of a list of $n$ departments, each department has
$n$ employees and $1$ manager, making a total of $n^{2} + n$ salaries.

    \item[geq] The `geq' benchmark compares two lists for equality, where the
               the lists are consecutive ranges $[1, n^{2}]$ and $[1, n^{2} +1]$ of
               integers.

    \item[min] The `min' benchmark calculates the minimum of a list.

    \item[sum] The `sum' benchmark calculates the sum of a list.
\end{description}

The `min' and `sum' tests should show the same performance, as both are the
same basic type of fold. They use the longer list of the `geq' benchmark.

Apparently, LIGD and EMGM perform roughly the same. Shapeless is slower in the
company benchmark, but faster in the other ones that only use lists which seems
to indicate that shapeless is more optimised for lists than LIGD and EMGM.

The HLIGD benchmark behaves like LIGD for the company benchmark, but is as fast
a the direct version for the other cases, because HLIGD directly uses
\cd{foldLeft} on sequences. Any time difference compared to the direct code is
noise.
