\section{Library overview}
\begin{table}[b]
  \centering
  \begin{tabular}{l|llll}
       & LIGD & EMGM & Uniplate & Shapeless \\
                 \hline
    Ad-hoc cases & \checkmark$^{+}$ & $\checkmark$ & \checkmark & \checkmark \\
    Multi-parameter functions & \checkmark & \checkmark & ($\times$) & \checkmark{} \\
    Extensibility & \checkmark $^{+}$ & $\circ$ & $\times$ & $\circ$ \\
    First-class generics & \checkmark & $\circ$ & $\circ$ & $\circ$ \\ \hline
    Automatic representations & $\times$ & $\times$ & ($\checkmark$) & \checkmark \\
    Ease of use & \checkmark & $\circ$ & \checkmark & $\circ$
  \end{tabular}
\begin{center}
\textbf{Legend:}\\
\begin{tabular}{llllll}
$\checkmark$ & Well supported & $\circ$ & Supported, but needs effort & $\times$ & Not supported \\
$^{+}$ & Not in Haskell & $()$ & Only tested in Haskell \\
\end{tabular}
\end{center}

  \label{library-diffs}
  \caption{Differences between the libraries}
\end{table}

The tested libraries support different sets of features. Table~\ref{library-diffs}
gives an overview of the differences. There are some more features that were
tested with the Haskell versions in GPBench~\cite{DBLP:conf/haskell/RodriguezJJGKO08},
but as they are supported by all of the libraries tested here, they are not
listed.

\paragraph{Ad-hoc cases}
All tested libraries support ad-hoc cases. In LIGD, only the Scala version
supports ad-hoc cases, due to the availability of sub-typing.

\paragraph{Multi-parameter functions}
Except for uniplate, all libraries support functions with multiple generic
arguments. Uniplate only has limited support for this, not sufficient to
implement generic equality, because it only looks at parts of objects that
have a specific type.

\paragraph{Extensibility}
With the class encoding of functions, LIGD supports
extensibility. EMGM supports extensibility out of the box, but is more verbose
to write than LIGD. Generic functions in Uniplate are not extensible, mostly
because they are just short functions based on combinators. In Shapeless, the
situation is similar, but there is one difference: the polymorphic functions
can be extended, allowing new ad-hoc cases.

\paragraph{First class generic functions}
LIGD does not directly support first class generic functions, they can be encoded as
objects with an apply method however. This is a limitation of the Scala
language, not a limitation of LIGD itself. See section~\ref{universal-types}
for a discussion on the issue. EMGM supports first class generics to some
extend, but is complicated. Uniplate does not support a really useful form
of first class generics. Shapeless mostly uses combinators, but
all generic functions are \cd{Poly} instances that can be passed around,
although using them requires some more implicit objects.

\paragraph{Automatic representations}
Only shapeless supports automatic representations. It should be possible to
automatically generate representations for Uniplate using Shapeless,
just like they can be generated in Haskell from \cd{Data} and \cd{Typeable}
instances, but this would require further investigation. It is also the slowest
form of instances Uniplate supports.

LIGD and EMGM have not been shown to support automatic generation of
representations yet. It might be possible to implement it using Scala
macros, but this requires further investigation as well.

\paragraph{Ease of use}
LIGD and Uniplate are very easy to learn and use: one is based on basic
pattern matching over representations and the other uses simple combinators
with monomorphic functions, both of which do not require the knowledge of
more advanced language features or library implementation details, even
when debugging.

EMGM is slightly less easy to use than LIGD. First of all, it requires more
code to achieve the same result; secondly, functions need to be extended to
sub-universes; and thirdly, writing the generic function in a visitor-like
way is going to be more complicated for most people.

Shapeless is very easy to use once the basics are understood. But Shapeless
also provides and uses more complex features. This often leads to error messages
that are hard to understand, making debugging a painful experience compared to
all of the other libraries.
