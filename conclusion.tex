\chapter{Conclusion}
Shapeless and LIGD are the best options for generic programming in Scala
right now. Shapeless provides a much larger feature set, is more
widely used, and should thus be preferred in production use. Where Uniplate
is applicable (like, for interpreters), it might make sense to use that
instead, possibly in combination with Shapeless for deriving instances like
done in Haskell using \cd{Data} and \cd{Typeable}.

EMGM does not seem to have any significant advantage compared to Shapeless that
can outweight its disadvantages in terms of verbosity.

The extended LIGD is now capable of supporting ad-hoc cases and extensibility,
making it more powerful then the original implementation in Haskell. Its use
of only basic language features and its easy-to-write representations and
functions make it an optimal choice for introducing generic programming to
students.

It would be interesting to look at automatic generation of LIGD representations
using macros; likewise, an implementation that represents constructors using
heterogeneous lists also seems a worthwhile idea, opening up new possibilities
like generic zippers. While an initial prototype of such a library exists in
the source code of this bachelor thesis\footnote{See \texttt{src/hligd.scala} and
\texttt{src/hlists.scala}~\cite{src}}, a more complete solution requires further
work.

Another topic to look at with regards to LIGD is the handling of lists and
other container types: Representing them as deeply-nested pairs can cause
stack overflows when implementing algorithms in a naïve way. Adding direct
support for sequences and/or implementing a set of combinators representing
commonly-needed operations using trampolines might be worthwhile.
