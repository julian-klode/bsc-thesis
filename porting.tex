\chapter{Porting from Haskell to Scala}

This chapter introduces some basic concepts for translating Haskell code
to Scala. Most of these concepts were explained by Oliveira and Gibbons~\cite{scalagp}, this
chapter summarizes them.

\section{Type classes and Instances}

Haskell has a concept of type classes and instances, whereas Scala offers
traits. Traits differ from type classes in that their methods always contain
an implicit \cd{this} object, that is, they combine data and functions;
whereas type classes do not reference any data.

\textbf{Question:} Given a type class like
\begin{lstlisting}[language=Haskell]
  class MyEq a where
    myEquals :: a -> a -> Boolean
    myEquals a b = False
\end{lstlisting}
how do we translate this to Scala?

\textbf{Answer:} There are two approaches, an object-oriented one and a more
functional-style one.

\subsection{Object-oriented translation}
The object-oriented approach uses traits just like interfaces in
Java are used. For example, the \cd{MyEq} type class would be translated
to the following trait:
\begin{lstlisting}
  trait MyEq[T] {
    def myEquals(b: T) : Boolean = false
  }
\end{lstlisting}
In order to implement this trait for a class the class must extend the
trait and override the methods of the trait:
\begin{lstlisting}
  class MyInt(val i: Int) extends MyEq[MyInt] {
    override def myEquals(b: MyInt) : Boolean = i == b.i
  }
\end{lstlisting}
This already shows one deficiency of the traits approach: We cannot
implement a trait for a type without modifying that type itself.

Oh the other hand, writing a function using this approach is straight forward:
\begin{lstlisting}
  def notEquals[T <: MyEq[T]](a: T, b: T) = !a.myEquals(b)
\end{lstlisting}

This style makes it easy to specify contexts, so for example, if we have
a class \cd{MyOrd} for ordering that also requires equality, we can write
\begin{lstlisting}
  trait MyOrd[T] extends MyEq[T] {
    def myLessOrEqual(b: T) : Boolean
  }
\end{lstlisting}

If we want to translate a type class containing a method that does not take a
parameter of the type \cd{T}, this style of writing becomes inconsistent.

\begin{example} Consider the type class
\begin{lstlisting}[language=Haskell]
  class PerformerFactory a where
    factor :: a
    perform :: a -> Boolean
\end{lstlisting}
We can only directly translate the \cd{perform} method to Scala. The \cd{factor}
method can not be expressed in this scheme, it would need to be a global
function for every type implementing \cd{PerformerFactory}. The translation
would thus be:
\begin{lstlisting}
  trait PerformerFactory {
    def perform: Boolean
  }
\end{lstlisting}
The end result is inconsistent, because one method is now a method of
instance objects whereas the other is translated to a global function. It
also permits instances of the trait that only implement the \cd{perform}
method.
\end{example}
\subsection{The functional approach}
Remember the type class MyEq:
\begin{lstlisting}[language=Haskell]
  class MyEq a where
    myEquals :: a -> a -> Boolean
    myEquals a b = False
\end{lstlisting}
We can ignore the implicit \cd{this} parameter in traits and translate the type
class as:
\begin{lstlisting}
  trait MyEq[A] {
    def myEquals(a: A, b: A) = false
  }
  def myEquals[T](a: T, b: T)(implicit eq: MyEq[T]) = eq.myEquals(a, b)
\end{lstlisting}
We also provided a global \cd{myEquals} function to make our life easier.

Now in order to implement the trait for a type, we need to define an instance
of the trait for this type. Instances of those traits correspond to dictionaries
in Haskell, and can be implemented like this:
\begin{lstlisting}
  implicit def MyEqInt = new MyEq[Int] {
    override def myEquals(a: Int, b: Int) = a == b
  }
\end{lstlisting}
The \cd{implicit} keyword is a special syntactic sugar provided by Scala
that makes our work easier. Without it, if we wanted to compare two objects,
we would need to manually pass the instance of the trait to the function,
like this:
\begin{lstlisting}
  > myEquals(1, 1)(MyEq)
  true
\end{lstlisting}
By making \cd{MyEqInt} and the \cd{eq} parameter of \cd{myEquals} implicit,
the Scala compiler can automatically infer that \cd{MyEqInt} is the
correct argument for the \cd{eq} parameter, and we can now run:
\begin{lstlisting}
  > myEquals(1, 1)
  true
\end{lstlisting}
We can also implement instances that require some sort of context. For
example, if we want to implement \cd{MyEq} for tuples of objects implementing
\cd{MyEq} as well, we can write:
\begin{lstlisting}
  implicit def myEqTuple[A,B](implicit ea: MyEq[A], eb: MyEq[B]) =
    new MyEq[(A,B)]{
    override def myEquals(a: (A,B), b: (A,B)) = ea.myEquals(a._1, b._1)
                                    && eb.myEquals(a._2, b._2)
  }
\end{lstlisting}
Now we can compare tuples: For example, in the following call, Scala will
automatically infer the correct instance of \cd{MyEq[(Int, Int)]}:
\begin{lstlisting}
  > myEquals((1,2), (1,2))
  true
\end{lstlisting}
We can still pass the dictionary manually as well:
\begin{lstlisting}
  > myEquals((1,2), (1,2))(myEqTuple(MyEqInt, MyEqInt))
  true
\end{lstlisting}

\section{Making the most of type inference}
\label{type-inference}

Compared to Haskell, Scala's type inference is weaker. For example, when
defining a function, the type of parameters need to be defined.

An exception to this are anonymous functions.
If the Scala compiler knows which type the
function must have, it can automatically infer the types of the parameters.

Thus, in order to avoid repetitious type information in the code,
we can change our \cd{MyEq} example to the one shown in listing~\ref{lambda}.
\begin{lstlisting}[caption=Improving type inference using lambdas,label=lambda]
  trait MyEq[A] {
    def myEquals : (A => A => Boolean) = a => b => false
  }
  implicit def MyEqInt = new MyEq[Int] {
    override def myEquals = a => b => a == b
  }
\end{lstlisting}
This way,
type annotations are not required in method implementations anymore. This
technique is shown in the port of the EMGM library.

% Discovered: 2014-08-04
Another thing that seems to matter a lot is the order of parameters. If
the less complex types are left of the more complex types, the more
complex types can be inferred more easily. For example, in Haskell,
\cd{foldl} has the signature:
\begin{lstlisting}[language=Haskell]
  foldl :: (a -> b -> a) -> a -> [b] -> a
\end{lstlisting}
Scala does not seem to be able to understand this, it will fail to
infer the type of the first argument. Moving the first argument to
the end fixes the issue, as Scala can then simply infer the types, from left
to right. It basically looks like this in Scala (except that
the first argument would actually be an object which foldLeft is a method
of):
\begin{lstlisting}
  foldLeft[a,b] : [b] => a => (a => b => a) => a
\end{lstlisting}

\section{Representing universal types}
\label{universal-types}
Haskell libraries often use universal types like:
\begin{lstlisting}[language=Haskell]
  forall a. Rep a => a -> String
\end{lstlisting}
Representing such types in Scala is a bit more complicated. We cannot use
functions -- even generic ones, because the Scala
compiler wants to bind the \cd{a} type as soon as we pass the function to
another function.

There are two alternative encodings, however. Both use the same encoding for
the functions, the difference is the encoding of the parameter type of a higher
order function.

So, first of all, in order to encode the a function of type \cd{forall a. Rep[a] => a -> b}
in Scala, we encode it as an object with an apply method containing the actual
function.
\begin{lstlisting}[caption=Encoding a universal function]
  object myfunction {
    def apply[A: Rep](a: A) : b = ...
  }
\end{lstlisting}
There are two variants now:

\subsection{Trait}
We can provide a  trait that specifies the type of function we want and make our
function object extend it. For example:
\begin{lstlisting}
  trait Returning[T] {
    def apply[A: Rep](a: A) : T
  }
\end{lstlisting}
Now we can create a higher order function accepting such a function (object)
by accepting an object of \cd{Returning[T]}.

\subsection{Structural type}
We can also use structural types to avoid the need of extending the trait, and
even the need of providing \cd{Returning}-like types for all number of parameters
we might need. We can still provide a \cd{Returning} structural type, if we like
though, so we do not have to write the complete type every time.

Using structural types the \cd{Returning} type looks like this:
\begin{lstlisting}
  type Returning[T] = { def apply[A: Rep](a: A) : T }
\end{lstlisting}
and every object providing such a method is implicitly an instance of that type.

Structural types have the minor disadvantage that \cd{scala.language.reflectiveCalls}
must be imported (or the equivalent command-line argument be set) when calling them.
