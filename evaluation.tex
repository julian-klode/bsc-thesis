\chapter{Evaluation of the approaches}
\todo{Lots of stuff to finish}
\section{Library overview}
\begin{table}[ht]
  \begin{tabular}{c|llll}
    Feature (and test)    & LIGD & EMGM & Uniplate & Shapeless \\
                 \hline
    Ad-hoc cases (company) & \checkmark$^{+}$ & $\checkmark$ & \checkmark & \checkmark \\
    Multi-parameter functions (geq) & \checkmark & \checkmark & ($\times$) & \checkmark{} \\
    Extensibility & \checkmark $^{+}$ & $\circ$ & ($\times$) & ($\times$) \\
    First-class generics & \checkmark & $\circ$ & ($\times$) & ($\times$)
  \end{tabular}


\begin{center}
\textbf{Legend:}\\
\begin{tabular}{llllll}
$\checkmark$ & Well supported & $\circ$ & Supported, but needs effort & $\times$ & Not supported \\
$^{+}$ & Not in Haskell & $()$ & Only tested in Haskell \\
\end{tabular}
\end{center}

  \caption{Feature overview}
\end{table}

The tested libraries support different sets of features. Of the tested libraries,
LIGD and EMGM were the most powerful, and LIGD was the easiest to write for.

\paragraph{Ad-hoc cases}
All tested libraries support ad-hoc cases. In LIGD, only the Scala version
supports ad-hoc cases, due to the availability of sub-typing.

\paragraph{Multi-parameter functions}
Except for uniplate, all libraries support functions with multiple generic
arguments. Uniplate only has limited support for this, not sufficient to
implement generic equality, because it only looks at parts of objects that
have a specific type.

\paragraph{Extensibility}
With the class encoding of functions, LIGD supports
With the class encoding of functions, LIGD supports
extensibility. EMGM supports extensibility out of the box, but is more verbose
to write than LIGD. \todo{Extensibility in Uniplate/Shapeless}

\section{Performance of the libraries}

To test the performance of the libraries, we ran various tests with them. For
each test, we executed it 600 thousand times, and calculated the average
time of the last 300 thousand executions.

The exact testing framework and all performed tests can be seen in
section~\ref{src:test}.

\begin{table}[ht]
\begin{tabular}{c|r|r|r|r}
\bf{test}  & \bf{company}    & \bf{geq}        & \bf{min}        & \bf{sum}       \\
\hline
Direct     &          3,0 μs &          0,4 μs &          0,6 μs &          0,4 μs\\
Shapeless  &         39,3 μs &             N/A &          4,7 μs &          4,8 μs\\
LIGD       &         16,0 μs &         15,3 μs &         13,5 μs &         13,8 μs\\
EMGM       &         18,4 μs &         16,5 μs &         12,2 μs &         16,5 μs\\
\end{tabular}

\caption{Benchmark results}
\label{bench}
\end{table}

As we can see in in table~\ref{bench}, all tested libraries perform roughly
the same time-wise.

\section{Final remarks}
