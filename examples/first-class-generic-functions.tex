\section{First class generic functions}

A generic function is a first class function if it can be passed to another
generic function. This requires support for existential types in the programming
language.

\subsection{LIGD}

Consider a generic function that can be applied to all representable values
and returns a string. For example, a function \cd{gshow} that shows values.
In Haskell, such a function would have the type:
\begin{lstlisting}[language=Haskell]
gshow :: forall a. Rep a -> a -> String
\end{lstlisting}

We want to represent this in Scala. Our first idea would be something like:
\begin{lstlisting}
gshow : (a => Rep[a] => String) forSome { type a }
\end{lstlisting}
This does not work however, the compiler will complain that the arguments do
not match the type a when calling the function. What we can do is make the
function a method of a trait, the trait being parameterized by the return
type and the method's type parameterized by the existential type. That is, the
Haskell type:
\begin{lstlisting}[language=Haskell]
gshow :: forall a. Rep a => a -> String
\end{lstlisting}
can be represented as \cd{Returning[String]}, where \cd{Returning} is defined
as:
\lstinputlisting[firstline=248,lastline=250,label=src:ligd:Returning,caption=Existential function in LIGD]{scala/src/ligd.scala}
Our \cd{gshow} becomes an instance of that trait, and then it is truly first-class (but looks ugly):
\lstinputlisting[firstline=255,lastline=271,label=src:ligd:Returning,caption=gshow in LIGD]{scala/src/ligd.scala}
We can now define another function accepting generic functions, for example,
\cd{gmapQ}:
\lstinputlisting[firstline=280,lastline=286,label=src:ligd:Returning,caption=gshow in LIGD]{scala/src/ligd.scala}
