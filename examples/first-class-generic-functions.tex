\section{First class generic functions}

A generic function is a first class function if it can be passed to another
generic function. This requires support for existential types in the programming
language.

\subsection{LIGD}

Consider a generic function that can be applied to all representable values
and returns a string. For example, a function \cd{gshow} that shows values.
In Haskell, such a function would have the type:
\begin{lstlisting}[language=Haskell]
gshow :: forall a. Rep a -> a -> String
\end{lstlisting}

Using the encoding of existential types shown in section~\ref{existential},
we can encode our gshow function as an object with an apply method:
\lstinputlisting[firstline=255,lastline=271,label=src:ligd:Returning,caption=gshow in LIGD]{scala/src/ligd.scala}
We can now define another function accepting generic functions, for example, \cd{gmapQ}:
\lstinputlisting[firstline=280,lastline=286,label=src:ligd:Returning,caption=gshow in LIGD]{scala/src/ligd.scala}

Note that if we encode the function as a class and an object extending that
class as shown in section~\ref{lidg-extensible}, we can even extend the
function with new special cases. As such, this encoding offers much more
flexibility compared to the function encoding, and there's not much reason
to use functions.
