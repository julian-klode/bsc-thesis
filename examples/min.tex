\section{Locating the smallest integer in a datatype}
As another example, let's take a look at what it takes to find the smallest
integer in a data structure.

\paragraph{LIGD}

\paragraph{EMGM} In EMGM, the definition is relatively straight forward, similar
to LIGD, although with a bit more boilerplate:

\lstinputlisting[firstline=318,lastline=337,caption=Minimum in EMGM,label=src:emgm:gmin]{scala/src/emgm.scala}

\paragraph{Uniplate}

Finding the minimum in Uniplate could look something like this:

\lstinputlisting[firstline=110,lastline=111,caption=Minimum in Uniplate,label=src:emgm:gmin]{scala/src/uniplate.scala}

That is, you find all integers in the data structure recursively, and then
simply pick the minimum. As a special bonus, this only works on data structures
than can contain integers, because the \cd{Biplate} has \cd{Int} as a second
parameter.

Of course, this needs adjustments for negative integers, but serves as a good
reference point. But to understand the idea, this is more than enough.
\paragraph{Shapeless}
