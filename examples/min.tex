\section{Locating the smallest integer in a datatype}
As another example, let's take a look at what it takes to find the smallest
integer in a data structure. In all cases, we return \cd{Int.MaxValue} if we cannot
find any (smaller) integers.

\paragraph{LIGD} The definition in LIGD is straight forward:
\lstinputlisting[firstline=288,lastline=295,caption=Minimum in LIGD,label=src:ligd:min]{scala/src/ligd.scala}

Alternatively, we can use one of the fold variants we implemented in our LIGD
port -- see the source code file \texttt{src/ligd.scala} \cite{src} for more examples.

\paragraph{EMGM} In EMGM, the definition is relatively straight forward, similar
to LIGD, although with a bit more boilerplate:

\lstinputlisting[firstline=295,lastline=314,caption=Minimum in EMGM,label=src:emgm:gmin]{scala/src/emgm.scala}

\paragraph{Uniplate}

Finding the minimum in Uniplate could look something like this:

\lstinputlisting[firstline=110,lastline=111,caption=Minimum in Uniplate]{scala/src/uniplate.scala}

That is, you find all integers in the data structure recursively, and then
simply pick the minimum. As a special bonus, this only works on data structures
than can contain integers, because the \cd{Biplate} has \cd{Int} as a second
parameter.
\paragraph{Shapeless}
Getting the minimum in shapeless works by using \cd{everything}. In the example
below, \cd{intmax} returns the maximum integer for non-integer values and the
integer for an integer value. The \cd{minimum} object than combines two integers
(it also does \cd{long}, \cd{float}, \cd{double}) using \cd{scala.math.min}.
\lstinputlisting[firstline=59,lastline=69,caption=Minimum in Shapeless,label=src:shapeless:min]{scala/src/shapeless.scala}
