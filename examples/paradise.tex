\section{Paradise benchmark}
GPBench is a suite of benchmarks for comparing generic programming
libraries in Haskell \cite{DBLP:conf/haskell/RodriguezJJGKO08}. One
benchmark included in it is the so-called paradise benchmark.

The paradise benchmark looks at data structures representing a company
and tries to transform it, by increasing every salary by 10\%.{}

\lstinputlisting[language=scala,firstline=28,lastline=39,float=ht,caption=Paradise benchmark data structures,label=src:data:company]{scala/src/data.scala}

The standard example data can be seen in Listing~\ref{src:data:company:example}.

\lstinputlisting[language=scala,firstline=41,lastline=60,float=ht,caption=Paradise benchmark example data,label=src:data:company:example]{scala/src/data.scala}

Apart from increasing the salary, we will also look at querying salaries.


\subsection{LIGD}

Given the data structures above and our implementation of LIGD that supports
subtyping \cd{RType}, implementing both operations is fairly straight forward.

First of all, the type representations. For the data types \cd{Person},
\cd{Employee}, and \cd{Company}, we can use the simplest form of definition,
by using the \cd{rType} function we added to our LIGD implementation.

\lstinputlisting[language=scala,firstline=29,lastline=31,float=ht,caption=Person\, Employee\, Company representation in LIGD,label=src:ligd-company:stuff]{scala/src/ligd-company.scala}

Because we want to query and transform salaries, we need to treat salaries
specially, so we can pattern match on it (otherwise we would need to compare,
but pattern matching is easier to read):

\lstinputlisting[language=scala,firstline=24,lastline=26,float=ht,caption=Salary representation in LIGD,label=src:ligd-company:salary]{scala/src/ligd-company.scala}

Finally, the \cd{DUnit} and \cd{Dept} types are mutually recursive. We would
like to use \cd{rType} here as well, so the compiler automatically infers more
stuff, but that implicitness breaks laziness and thus cause an endless
recursion. We can preserve laziness by using the \cd{RType} constructor instead
of our short-hand \cd{rType} function and specifying the (recursive) first
parameter manually \footnote{We could even use the \cd{rep} function we added to
LIGD here, but that would decrease readability}.

\lstinputlisting[language=scala,firstline=34,lastline=43,float=t,caption=Mutually recursive representations,label=src:ligd-company:mutual]{scala/src/ligd-company.scala}

Now let us look at two examples.

\begin{example}[Query: Summing all salaries]

Querying salaries is relatively straight forward. Pattern match on all standard
LIGD types, and recursive for the product, sum, RType types; but before matching
against \cd{RType} match against \cd{RSalary}:

\lstinputlisting[language=scala,firstline=46,lastline=55,caption=Summing salaries in LIGD (manually),label=src:ligd-company:sumSalaryOld]{scala/src/ligd-company.scala}
\end{example}

\subsubsection*{Example 2: Transformation: Increase all salaries}




\subsection{EMGM}

\subsection{Uniplate}

\subsection{Shapeless}
