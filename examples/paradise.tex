\section{Paradise benchmark}
GPBench, developed by Rodriguez et. al., is a suite of benchmarks for comparing
generic programming libraries in Haskell~\cite{DBLP:conf/haskell/RodriguezJJGKO08}.
One benchmark included in it is the so-called paradise benchmark.

The paradise benchmark looks at data structures representing a company
and tries to transform it, by increasing every salary by 10\%.{}

\lstinputlisting[firstline=28,lastline=39,float=ht,caption=Paradise benchmark data structures,label=src:data:company]{scala/src/data.scala}

The standard example data can be seen in Listing~\ref{src:data:company:example}.

\lstinputlisting[firstline=41,lastline=50,float=ht,caption=Paradise benchmark example data,label=src:data:company:example]{scala/src/data.scala}

Apart from increasing the salary, we will also look at querying salaries.


\subsection{LIGD}

Given the data structures above and our implementation of LIGD that supports
subtyping \cd{RType}, implementing both operations is fairly straight forward.

First of all, the type representations. For the data types \cd{Person},
\cd{Employee}, and \cd{Company} we can use the simplest form of definition,
by using the \cd{rType} function we added to our LIGD implementation.

\lstinputlisting[firstline=29,lastline=30,frame=t,caption={Person, Employee, and Company representations in LIGD},label=src:ligd-company:stuff,style=breaklines]{scala/src/ligd-company.scala}
\lstinputlisting[firstline=46,lastline=46,frame=b]{scala/src/ligd-company.scala}

Because we want to query and transform salaries, we need to treat salaries
specially, so we can pattern match on it (otherwise we would need to compare,
but pattern matching is easier to read):

\lstinputlisting[firstline=24,lastline=26,float=ht,caption=Salary representation in LIGD,label=src:ligd-company:salary]{scala/src/ligd-company.scala}

Finally, the \cd{DUnit} and \cd{Dept} types are mutually recursive. Simply
using \cd{rType} here as we have done for the other types would not work due
to two reasons: First, the Scala compiler would not be able to infer the types;
and second, because the implicit \cd{Rep} of \cd{rType} is not lazy, there would
be either endless recursion (if both were defined using \cd{def}) or \cd{rDept}
would be \cd{null} in the definition of \cd{rDUnit} because \cd{rDept} is defined
after \cd{rDUnit} and thus not yet initialized.

Fixing this is simple: for the first issue, we explicitly specify the types of
the representations. For the second issue, we can simply define \cd{rDUnit} in
terms of \cd{RType} instead of \cd{rType}. This makes the reference to \cd{rDept}
lazy and thus allows it to be used here.

\lstinputlisting[firstline=34,lastline=43,float=ht,caption=Mutually recursive representations,label=src:ligd-company:mutual]{scala/src/ligd-company.scala}

\begin{example}[Query: Summing all salaries]

Querying salaries is relatively straight forward. Pattern match on all standard
LIGD types, and recursive for the product, sum, and \cd{RType} types; but before matching
against \cd{RType} match against \cd{RSalary}:

\lstinputlisting[firstline=49,lastline=59,caption=Summing salaries in LIGD (manually),label=src:ligd-company:sumSalaryOld]{scala/src/ligd-company.scala}

We can, however, also use the generic fold we implemented earlier, the definition
then becomes a single line of code:

\lstinputlisting[firstline=70,lastline=70,caption=Summing salaries in LIGD (fold),label=src:ligd-company:sumSalary]{scala/src/ligd-company.scala}

Note: If we swapped around the first two parameters of \cd{gfoldl} we would not even
need to specify the \cd{Float} type, as long as we would pass \cd{0.0} instead
of 0.

\end{example}

\begin{example}[Transformation: Increase all salaries]

Increasing the salaries by 10\% is what is usually called the paradise
benchmark. Implementing this is straight forward as well in LIGD:

\lstinputlisting[firstline=61,lastline=68,caption=Increasing the salaries in LIGD,label=src:ligd-company:incSalary]{scala/src/ligd-company.scala}

We can abstract this by building an \cd{everywhere} function\footnote{a sample implementation of \cd{everywhere} is included in the source code} that applies
a transformation function, similar to EMGM and Uniplates \cd{transformBi} -- and
then calling it with a salary-transforming function on our company,
like this:
\begin{lstlisting}[caption=Increase the salaries using \cd{everywhere}]
  everywhere((i: Salary) ⇒ Salary(i.salary * 1.1F))(genCom)
\end{lstlisting}

\end{example}

In any case, both examples are easy to implement.


\subsection{EMGM}

Implementing the paradise benchmark in EMGM requires more boilerplate compared
to LIGD. This version is slightly more powerful than the LIGD variant (it
provides ad-hoc cases for all types in a company, compared to only \cd{Salary}),
but only because this makes the code more readable.

We first start by extending \cd{Generic} with new methods for all types of
company objects. This allows us to write functions with ad-hoc cases for all
of those types.

\lstinputlisting[firstline=38,lastline=58,caption=GenericCompany,label=src:emgm-company:GenericCompany,style=breaklines]{scala/src/emgm-company.scala}

Now we need some boilerplate to have implicit \cd{GRep} instances, as shown
in listing~\ref{src:emgm-company:GRep}.
\lstinputlisting[firstline=61,lastline=75,caption=Implicit instances of \cd{GRep},label=src:emgm-company:GRep]{scala/src/emgm-company.scala}

\begin{example}[Query: Summing all salaries]
Summing all salaries works as expected by using a very simple (although
verbose) definition:
\lstinputlisting[firstline=99,lastline=116,caption=Summing salaries in EMGM,label=src:emgm-company:sumSalary]{scala/src/emgm-company.scala}
\end{example}

\clearpage
\begin{example}[Transformation: Increase all salaries]
It's straight forward to implement this in EMGM, although the definition is
very verbose.
\lstinputlisting[firstline=78,lastline=96,caption=Using the transform pattern: Increase salary,label=src:emgm-company:incSalary]{scala/src/emgm-company.scala}
\end{example}

\subsection{Uniplate}
Uniplate is applicable here, and as promised in chapter 3, it has the shortest
solutions, due to its unique approach.\footnote{Not counting type representations}

\begin{example}[Query: Summing all salaries]
 Trivial to do in uniplate: Get all salaries and fold over them:
\begin{lstlisting}
  def sumSalary[C](self: C)(implicit bp: Biplate[C, Salary])
    = universeBi(self).foldLeft(0.0F)((f, s) => f + s.salary)
\end{lstlisting}
\end{example}

\begin{example}[Transformation: Increase all salaries]
 A transformation is done using the \cd{transformBi} function which takes a function of
 type \cd{A => A} and applies it to a value of a type \cd{B} using a \cd{Biplate[B,A]}.
\begin{lstlisting}
  transformBi((i: Salary) ⇒ Salary(i.salary * 1.1F))(genCom)
\end{lstlisting}
\end{example}


\subsection{Shapeless}

Using shapeless is much easier than using our self-created libraries because
there is no need to create type representations manually, shapeless takes care
of that for us.

\begin{example}[Query: Summing all salaries]
We can simply use the \cd{everything} combinator presented in chapter 3.

\lstinputlisting[firstline=75,lastline=83,caption=Summing salaries in Shapeless]{scala/src/shapeless.scala}

To recall, the \cd{everything} combinator consists of two parts: A traversal function,
here \cd{extractSalary}, that extracts the types we are interested in,
and a function to combine the results from the traversal, such as the
\cd{addFloat} here.
\end{example}

\begin{example}[Transformation: Increase all salaries]

Transformations like that are easy to write in Shapeless. All we need to do
is create a function that takes a \cd{Salary} and produces a new one and pass
that to Shapeless' \cd{everywhere} function.

\lstinputlisting[firstline=67,lastline=68,caption=Increasing salaries in Shapeless]{scala/tests/CompanyTest.scala}

\end{example}
