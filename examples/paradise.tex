\section{Paradise benchmark}
GPBench is a suite of benchmarks for comparing generic programming
libraries in Haskell \cite{DBLP:conf/haskell/RodriguezJJGKO08}. One
benchmark included in it is the so-called paradise benchmark.

The paradise benchmark looks at data structures representing a company
and tries to transform it, by increasing every salary by 10\%.{}

\lstinputlisting[language=scala,firstline=28,lastline=39,float=ht,caption=Paradise benchmark data structures,label=src:data:company]{scala/src/data.scala}

The standard example data can be seen in Listing~\ref{src:data:company:example}.

\lstinputlisting[language=scala,firstline=41,lastline=60,float=ht,caption=Paradise benchmark example data,label=src:data:company:example]{scala/src/data.scala}

Apart from increasing the salary, we will also look at querying salaries.


\subsection{LIGD}

Given the data structures above and our implementation of LIGD that supports
subtyping \cd{RType}, implementing both operations is fairly straight forward.

First of all, the type representations. For the data types \cd{Person},
\cd{Employee}, and \cd{Company}, we can use the simplest form of definition,
by using the \cd{rType} function we added to our LIGD implementation.

\lstinputlisting[language=scala,firstline=29,lastline=31,float=ht,caption=Person\, Employee\, Company representation in LIGD,label=src:ligd-company:stuff]{scala/src/ligd-company.scala}

Because we want to query and transform salaries, we need to treat salaries
specially, so we can pattern match on it (otherwise we would need to compare,
but pattern matching is easier to read):

\lstinputlisting[language=scala,firstline=24,lastline=26,float=ht,caption=Salary representation in LIGD,label=src:ligd-company:salary]{scala/src/ligd-company.scala}

Finally, the \cd{DUnit} and \cd{Dept} types are mutually recursive. We would
like to use \cd{rType} here as well, so the compiler automatically infers more
stuff, but that implicitness breaks laziness and thus cause an endless
recursion. We can preserve laziness by using the \cd{RType} constructor instead
of our short-hand \cd{rType} function and specifying the (recursive) first
parameter manually \footnote{We could even use the \cd{rep} function we added to
LIGD here, but that would decrease readability}.

\lstinputlisting[language=scala,firstline=34,lastline=43,float=t,caption=Mutually recursive representations,label=src:ligd-company:mutual]{scala/src/ligd-company.scala}

Now let us look at two examples.

\begin{example}[Query: Summing all salaries]

Querying salaries is relatively straight forward. Pattern match on all standard
LIGD types, and recursive for the product, sum, RType types; but before matching
against \cd{RType} match against \cd{RSalary}:

\lstinputlisting[language=scala,firstline=46,lastline=55,caption=Summing salaries in LIGD (manually),label=src:ligd-company:sumSalaryOld]{scala/src/ligd-company.scala}

We can however also use the generic fold we implemented earlier, the definition
becomes a single line of code then:

\lstinputlisting[language=scala,firstline=67,lastline=67,caption=Summing salaries in LIGD (fold),label=src:ligd-company:sumSalary]{scala/src/ligd-company.scala}

Note: If we swapped around the first two parameters of \cd{gfoldl} we would not even
need to specify the \cd{Float} type, as long as we would pass \cd{0.0} instead
of 0.

\end{example}

\begin{example}[Transformation: Increase all salaries]

Increasing the salaries by 10\% is what is usually called the paradise
benchmark. Implementing this is straight forward as well in LIGD:

\lstinputlisting[language=scala,firstline=58,lastline=65,caption=Increase the salaries in LIGD,label=src:ligd-company:sumSalary]{scala/src/ligd-company.scala}

You can abstract this by building an \cd{everywhere} function\footnote{a sample implementation of \cd{everywhere} is included in the source code} that applies
a transformation function, similar to EMGM and Uniplates \cd{transformBi} -- then
you can just call everywhere with a salary-transforming function on your company,
like this:
\begin{lstlisting}[caption=Increase the salaries using \cd{everywhere}]
  everywhere((i: Salary) ⇒ Salary(i.salary * 1.1F))(genCom)
\end{lstlisting}

\end{example}

In any case, both examples are straight forward to implement. It cannot get
much better.


\subsection{EMGM}

Implementing the paradise benchmark in EMGM requires more boilerplate compared
to LIGD. This version is slightly more powerful than the LIGD variant (it
provides ad-hoc cases for all types in a company, compared ton only \cd{Salary}),
but only because this makes the code more readable.

We first starting by extending \cd{Generic} with new methods for all types of
company objects. This allows us to write functions with ad-hoc cases for all
of those types.

\lstinputlisting[language=scala,firstline=38,lastline=58,caption=GenericCompany,label=src:emgm-company:GenericCompany]{scala/src/emgm-company.scala}

Now we need some boilerplate to have implicit \cd{GRep} instances:
\lstinputlisting[language=scala,firstline=61,lastline=75,caption=Implicit instances of \cd{GRep},label=src:emgm-company:GRep]{scala/src/emgm-company.scala}



\begin{example}[Query: Summing all salaries]

\end{example}

\begin{example}[Transformation: Increase all salaries]

\end{example}

\subsection{Uniplate}


\begin{example}[Query: Summing all salaries]

\end{example}

\begin{example}[Transformation: Increase all salaries]

\end{example}


\subsection{Shapeless}


\begin{example}[Query: Summing all salaries]

\end{example}

\begin{example}[Transformation: Increase all salaries]

Transformations like that are easy to write in Shapeless. All we need to do
is create a function that takes a \cd{Salary} and produces a new one and pass
that to Shapeless' \cd{everywhere} function.

\lstinputlisting[firstline=63,lastline=64,caption=Increasing salaries in Shapeless]{scala/tests/CompanyTest.scala}

\end{example}
