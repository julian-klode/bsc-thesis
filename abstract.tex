\begingroup
\let\titlepage\par
\let\endtitlepage
\let{}
\cleardoublepage{}
\thispagestyle{empty}
\begin{otherlanguage}{ngerman}
\begin{abstract}
Generische Programmierung ist eine Vorgehensweise, die es ermöglicht, Funktionen
über die Struktur eines Datentypes zu definieren, anstelle eines spezifischen
Datentypes. Dies erlaubt es einer Funktion auf beliebigen Datentypen zu operieren,
selbst auf solchen, die dem Programmier der Funktion noch nicht bekannt waren.

In dieser Bachelorarbeit werde ich Portierungen von drei Haskell Bibliotheken
(LIGD, EMGM, Uniplate) nach Scala vorstellen und ihre Funktionalität und Geschwindigkeit miteinander
, sowie mit Shapeless, vergleichen.

Ich werde außerdem zeigen, dass LIGD in Scala erweitert werden kann, sodass sie
beinahe so mächtig wie EMGM ist, indem ich das Problem der Erweiterbarkeit
von Funktionen löse, welches bisherige Implementierungen stark beeinträchtigte.
\end{abstract}
\end{otherlanguage}
\begin{abstract}
Generic programming is a technique that allows functions to be defined over
the shape of an type, rather than a specific type; thus allowing one function
to be used on objects of arbitrary types, even those that were not known to
the programmer of the function.

In this thesis, I will present ports of three Haskell libraries (LIGD, EMGM, Uniplate) to Scala,
and compare their features and performance against each other and
against shapeless, a native Scala library.

I will show that LIGD in Scala can be extended to be almost as powerful as EMGM,
solving the issue with extensibility that made it vastly inferior in previous
implementations.


\end{abstract}
\endgroup
